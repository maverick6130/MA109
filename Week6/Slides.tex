\documentclass[aspectratio=169]{beamer}

\mode<presentation>{}
\usepackage[utf8]{inputenc}
\usepackage{setspace}
\newcommand{\fl}[1]{\left\lfloor #1 \right\rfloor}
\setstretch{1.25}
\usepackage{tikz}
\title{MA109 Tutorial Session}
\subtitle{Week 6}
\author{Dhruv Arora}
\institute{Sophomore, Dept of CSE}
\date{\today}
\usetheme{CambridgeUS}
\usecolortheme{seahorse}


\newtheorem{thm}{Theorem}
\newtheorem{cor}[thm]{Corollary}
\newtheorem{df}{Definition}
\newtheorem{qsn}{Question}
\newcommand{\bZ}{\mathbb{Z}}
\newcommand{\bN}{\mathbb{N}}
\newcommand{\bR}{\mathbb{R}}
\newcommand{\norm}[1]{|#1|}
\newcommand{\normtwo}[1]{||#1||}
\begin{document}
\begin{frame}
	\titlepage
	\begin{center}
	{\tiny with most (all) of the effort from}\\
	{\small Harshit Gupta}\\
	{\tiny Sophomore, Dept of CSE}
	\end{center}
\end{frame}

\begin{frame}{Summary} 
Tutorial Sheet 6:  2,4,5,8,9
\end{frame}

\begin{frame}{Sheet 6, Q2}
	\begin{qsn}
        Find the directions in which the directional derivative of $f(x, y) = x^2 + \sin xy$ at the point
        $(1, 0)$ has the value 1.
	\end{qsn}
	\pause
	Since the above function is a combination of polynomial and sinusoidal functions, $f(x,y)$ is continuous as well as differentiable in $\mathbb{R}^2$.\\
	\pause
	The directional derivative can be calculated using the total derivative. The total derivative at point $(1,0)$ is given by
	$$
	    \nabla f(1,0)  = \left( \frac{\partial f}{\partial x}(1,0) , \frac{\partial f}{\partial y} (1,0)\right) = \left( 2, 1\right)
	$$
\end{frame}

\begin{frame}{Sheet 6, Q2}
	The directional derivative in the unit direction $u = (u_1,u_2)$ is given by $\mathbf{D_u}f(1,0) =  \nabla f(1,0) \cdot u = 2u_1 + u_2$.\\
	\pause
	We have to solve the following equations: 
	\begin{align*}
	2u_1 + u_2 = 1\\
	 u_1^2 + u_2^2 = 1\\
	\end{align*}
	\\\vspace{-20pt}
	\pause
	This gives us $(u_1, u_2) = (0,1)$ or $(0.8,-0.6)$. These are the required directions. 
\end{frame}
\begin{frame}{Sheet 6, Q4}
	\begin{qsn}
    Find $\mathbf{D_u}F(2, 2, 1)$, where $F(x, y, z) = 3x - 5y + 2z$, and $u$ is the unit vector in the direction of the outward normal to the sphere $g(x,y,z) = x^2 + y^2 + z^2 - 9 = 0$ at (2, 2, 1).
    \end{qsn}
    \uncover<2-> {
    As covered in class, the normal for a function $f(x,y,z)=0$ is given by its gradient. So, 
    $$
        u = \frac{\nabla g(2,2,1)}{||\nabla g(2,2,1)||} = \left(\frac{2}{\sqrt{2^2+2^2+1^2}},\frac{2}{\sqrt{2^2+2^2+1^2}},\frac{1}{\sqrt{2^2+2^2+1^2}}\right) = \left(\frac{2}{3},\frac{2}{3},\frac{1}{3}\right)
    $$
	}
    \uncover<3-> {
    By the definition of direction derivative, we get
    $$
    (\mathbf{D_u}F)(2, 2, 1) =  \lim_{t\to 0} \frac{F(ut+(2,2,1)) - F(2,2,1)}{t} = \lim_{t\to 0}\frac{3(2t/3) - 5(2t/3) + 2(t/3)}{t} = -\frac{2}{3}
    $$
	}
\end{frame}

\begin{frame}{Sheet 6, Q5}
    \begin{qsn}
        Given $\sin (x+y) + \sin (y+z) = 1$, find $\frac{\partial^2 z}{\partial x \partial y}$ provided $\cos (y+z) \ne 0$.
    \end{qsn}
    \uncover<2->{
    % We shall assume that $z$ is a ``sufficiently smooth'' function of $x$ and $y.$\\
	We are given that $\sin (x+y)+\sin (y+z)=1$ and $\cos (y+z) \neq 0.$\\ 
	}
    \uncover<3-> {
    Differentiating with respect to $x$ while keeping $y$ constant gives us $\cos (x+y)+\cos (y+z) \frac{\partial z}{\partial x}=0.$ \hfill $(*)$\\
		}
    \uncover<4-> {
    Similarly, differentiating with respect to $y$ while keeping $x$ constant gives us $\cos (x+y)+\cos (y+z)\left(1+\frac{\partial z}{\partial y}\right)=0.$ \hfill $(**)$\\
		}

\end{frame}

\begin{frame}{Sheet 6, Q5}
    Differentiating $(*)$ with respect to $y$ gives us $-\sin (x+y)-\sin (y+z)\left(1+\frac{\partial z}{\partial y}\right) \frac{\partial z}{\partial x}+\cos (y+z) \frac{\partial^{2} z}{\partial x \partial y}=0.$
    \pause
    \\Thus, using $(*)$ and $(**),$ we get
	\begin{align*} 
	\frac{\partial^{2} z}{\partial x \partial y} &=\frac{1}{\cos (y+z)}\left[\sin (x+y)+\sin (y+z) \cdot\left(1+\frac{\partial z}{\partial y}\right) \frac{\partial z}{\partial x}\right] \\[10pt]
	&=\frac{1}{\cos (y+z)}\left[\sin (x+y)+\sin (y+z)\left(-\frac{\cos (x+y)}{\cos (y+z)}\right)\left(-\frac{\cos (x+y)}{\cos (y+z)}\right)\right] 
	\\[10pt] &=\frac{\sin (x+y)}{\cos (y+z)}+\tan (y+z) \frac{\cos ^{2}(x+y)}{\cos ^{2}(y+z)} \end{align*}
\end{frame}

\setstretch{1}
\begin{frame}{Sheet 6, Q8}
    \begin{qsn}
        Analyse the following functions for local maxima, local minima and saddle points:\\
        (i) $f(x,y) = (x^2 - y^2) e^{-(x^2+y^2)/2}$ \hspace{15pt} 
    \end{qsn}
    \uncover<2->{Note that the above function is defined on $Domain = \mathbb{R}^2.$}\\
	\uncover<3->{Thus, every point is an interior point of $Domain.$ Moreover, it can be seen that the partial derivatives of all orders exist and are continuous everywhere.\\ }
	\uncover<4->{For $(x_0, y_0)$ to be a point of extrema or a saddle point, it must be the case that $(\nabla f)(x_0, y_0) = (0, 0).$ }\\
	\uncover<5->{Note that $f_x(x, y) =x e^{1 / 2\left(-x^{2}-y^{2}\right)}\left(-x^{2}+y^{2}+2\right).$ }\\
	\uncover<6->{Also, $f_y(x, y) =y e^{1 / 2\left(-x^{2}-y^{2}\right)}\left(-x^{2}+y^{2}-2\right).$ }\\
	\uncover<7->{solving $(\nabla f)(x_0, y_0) = (0, 0)$ gives us that $(x_0, y_0) \in \{(0, 0),\;(0, \sqrt{2}),\;(0, -\sqrt{2}),\;(-\sqrt{2},0),\;(\sqrt{2}, 0)\}.$ }\\
	\uncover<8->{Now, we determine the exact nature using the second derivative test. }
\end{frame}
\setstretch{1.25}

\begin{frame}{Sheet 6, Q8}
	Recall that $D:=f_{x x}\left(x_{0}, y_{0}\right) f_{y y}\left(x_{0}, y_{0}\right)-f_{x y}\left(x_{0}, y_{0}\right)^{2}.$\\
	\uncover<2->{Hence, in our case,
	\vspace{-15pt}
	\[D = -e^{-x^{2}-y^{2}}\left(x^{6}-x^{4} y^{2}-3 x^{4}-x^{2} y^{4}+22 x^{2} y^{2}-8 x^{2}+y^{6}-3 y^{4}-8 y^{2}+4\right).\] }
	\\\vspace{-20pt}
	\uncover<3->{Moreover, $f_{xx}(x, y) = e^{-\left(x^{2}+y^{2}\right) / 2}(x^4 - x^2y^2 - 5x^2 + y^2 + 2)$ }\\
	\uncover<3->{For $(x_0, y_0) = (0, 0),$ it is clear that it is a saddle point for $f$ as discriminant is $-4 < 0.$ }\\
	\uncover<4->{Note that if $x = 0,$ the discriminant reduces to $-e^{-y^2}(y^6 - 3y^4 -8y^2 + 4).$ }\\
	\uncover<5->{Substituting $y = \pm\sqrt{2}$ gives us that the discriminant is positive with $f_{xx}$ positive and hence, the points are points of local minima. }\\
	\uncover<6->{Similarly, we get that the points $(\pm\sqrt{2}, 0)$ are points of local maxima as they have discriminant positive and $f_{xx}$ negative. }
\end{frame}
\begin{frame}{Sheet 6, Q8}
   \begin{qsn}
        Analyse the following functions for local maxima, local minima and saddle points:\\
        (ii) $ f(x,y) = x^3 - 3xy^2 $
   \end{qsn}
   	\uncover<2->{Note that the above function is defined on $Domain = \mathbb{R}^2.$}\\
	\uncover<3->{Thus, every point is an interior point of $Domain.$ Moreover, it can be seen that the partial derivatives of all orders exist and are continuous everywhere. \\}
	\uncover<4->{For $(x_0, y_0)$ to be a point of extrema or a saddle point, it must be the case that $(\nabla f)(x_0, y_0) = (0, 0).$ }\\
	\uncover<5->{Note that $f_x(x, y) = 3x^2 - 3y^2.$ }\\
	\uncover<6->{Also, $f_y(x, y) = -6xy.$ }\\
	\uncover<7->{Thus, solving $(\nabla f)(x_0, y_0) = (0, 0)$ gives us that $(x_0, y_0) = (0, 0).$ }\\
	\uncover<8->{Now, we determine the exact nature using the second derivative test. }
\end{frame}
\begin{frame}{Sheet 6, Q8}
    Recall that $D:=f_{x x}\left(x_{0}, y_{0}\right) f_{y y}\left(x_{0}, y_{0}\right)-f_{x y}\left(x_{0}, y_{0}\right)^{2}.$\\
	\uncover<2->{Hence, in our case,
	\[D(x_0, y_0) = -36(x_0^2 + y_0^2).\] }
	\uncover<3->{Thus, for $(x_0, y_0) = (0, 0),$ we get the discriminant is $0.$}\\
	\uncover<4->{Hence, we get that }\uncover<5->{the discriminant test is {\color[rgb]{1, 0, 0} inconclusive!} }\\
	\uncover<6->{This means that we must turn to some other analytic methods of determining the nature. }\\[10pt]
	\uncover<7->{Note that $f(\delta, 0) = \delta^3$ for all $\delta \in \mathbb{R}.$ }\\
	\uncover<8->{Take any $r > 0$. Take the points $(r/2,0)$ and $(-r/2,0)$. For these, $||(x,y)||<r$ but $f(-r/2,0)<f(0,0)<f(r/2,0)$}\\
	\uncover<9->{This gives us that $(0, 0)$ is saddle point. }
\end{frame}
\begin{frame}{Sheet 6, Q9}
    \begin{qsn}
        Find the absolute maximum and absolute minimum of $f(x,y) = (x^2-4x) \cos y$ for $1\leq x \leq 3$, $-\pi/4 \leq y \leq \pi/4$
    \end{qsn}
    \uncover<2->{Note that the domain is a closed and bounded set. As $f$ is continuous on the domain, $f$ does achieve a maximum and a minimum. Further since $f(x,y) \in C^\infty$, the points where it can achieve the maximum and minimum are the critical points, and the boundary.\\}
	\uncover<3->{ Note that $f_x(x, y) = (2 x-4) \cos y$ and $f_y(x, y) = -\left(x^{2}-4 x\right) \sin y$ for interior points $(x, y).$}\\
	\uncover<4->{Thus, the only critical point is $p_1 = (2, 0).$}\\
	
\end{frame}
\begin{frame}{Sheet 6, Q9}
    \uncover<1->{Now we restrict ourselves to the boundaries to find the local extrema.}\\
	\uncover<2->{``Right boundary:'' This is the line segment $x = 3, -\pi / 4 \leq y \leq \pi / 4.$ }\\
	\uncover<3->{The function now reduces to $-3\cos y$ on this segment. }\\
	\uncover<4->{Using our theory from one-variable calculus, we get that we need to check the points $(3, 0),\;(3, \pi/4),\;(3, -\pi/4).$ } \hfill \uncover<7->{(How?)}\\
	\uncover<5->{Similarly, considering the ``left boundary'' gives the points $(1, 0),\;(1, \pi/4),\;(1, -\pi/4).$}\\
	\uncover<5->{Now, we look at the ``top boundary.''. The function there reduces to $\frac{x^2 - 4x}{\sqrt{2}}.$ }\\
	\uncover<6->{Once again, using our theory from one-variable calculus, we get that we need to check the points $(1, \pi/4),\;(2, \pi/4),\;(3, \pi/4).$ }\\
	\uncover<7->{Similarly, checking the ``bottom boundary'' gives us the points $(1, -\pi/4),\;(2, -\pi/4),\;(3, -\pi/4).$ }\\
\end{frame}
\begin{frame}{Sheet 6, Q9}
    \[\begin{array}{|c||c|c|c|c|c|}
	\hline
	(x_0, y_0) & (2, 0) & (3, 0) & (3, \pi/4) & (2, \pi/4) & (1, \pi/4) \\
	\hline
	f(x_0, y_0) & -4 & -3 & \dfrac{-3}{\sqrt{2}} & \dfrac{-4}{\sqrt{2}} & \dfrac{-3}{\sqrt{2}} \\
	\hline
	\hline
	(x_0, y_0) & (1, 0) & (1, -\pi/4) & (2, -\pi/4) & (3, -\pi/4) &  \\
	\hline
	f(x_0, y_0) & -3 & \dfrac{-3}{\sqrt{2}} & \dfrac{-4}{\sqrt{2}} & \dfrac{-3}{\sqrt{2}} & \\
	\hline 
	\end{array}
	\]
	\uncover<2->{Thus, we get that $f_{\min} = -4$ at $(2, 0)$ and $f_{\max} = -\frac{3}{\sqrt{2}}$ at $(1, \pm \pi/4)$ and $(3, \pm\pi/4).$}
\end{frame}
\end{document}