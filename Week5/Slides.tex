\documentclass[aspectratio=169]{beamer}

\usepackage{amsmath, amsthm, amssymb, esint}
\usetheme{CambridgeUS}
\usecolortheme{seahorse}

\renewcommand\qedsymbol{$\spadesuit$}

\title{MA109 Tutorial Session}
\subtitle{Week 5}
\author{Dhruv Arora}
\institute{Sophomore, Dept of CSE}
\date{\today}

\newtheorem{thm}{Theorem}
\newtheorem{cor}[thm]{Corollary}
\newtheorem{df}{Definition}
\newtheorem{qsn}{Question}

\newcommand{\bZ}{\mathbb{Z}}
\newcommand{\bN}{\mathbb{N}}
\newcommand{\bR}{\mathbb{R}}
\newcommand{\bQ}{\mathbb{Q}}
\newcommand{\norm}[1]{|#1|}
\newcommand{\normtwo}[1]{||#1||}

\begin{document}
\begin{frame}
	\titlepage
	\begin{center}
	{\tiny with most of the effort from}\\
	{\small Harshit Gupta}\\
	{\tiny Sophomore, Dept of CSE}
	\end{center}
\end{frame}

\begin{frame}[plain]
\frametitle{What's New this Wednesday}
\tableofcontents
\end{frame}

\section{Tutorial Sheet 5}

\subsection{Q2. Contour Lines and Level Curves
}

\begin{frame}{Recap}
	\uncover<2-> {
	\begin{df}[Level Curve]
		Let $f:\bR^2 \to \bR$ be a function and $c\in\bR$. Then the set $\{(x,y)\in\bR^2 | f(x,y)=c\}\subseteq \bR^2$ is called the level curve of $f$ corresponding to $c$.
	\end{df}
	}
	\uncover<3-> {
	\begin{df}[Contour Line]
		Let $f:\bR^2 \to \bR$ be a function and $c\in\bR$. Then the set $\{(x,y,c)\in\bR^3 | f(x,y)=c\}\subseteq \bR^3$ is called the contour line of $f$ corresponding to $c$.
	\end{df}
	}
	\uncover<4-> {
	The difference is subtle. Simply stated, $\mathcal{C} = \mathcal{L}\times\{c\}$\\ 
	}
	\uncover<5-> {
	Thus, I will only cover level curves in the following question.
	}
\end{frame}

\begin{frame}{Caution}
	\begin{block}{Note}
	\uncover<2-> {
	The ``curve" and ``line" in level curve and contour line is in not to be taken literally!
	}
	\uncover<3-> {
	A level curve only needs to be a subset of $\bR^2$. Infact, you will see these mentioned as ``level sets" and ``contour plots" in different books.
	}
	\end{block}
	\uncover<4-> {
	\vspace*{2mm}
	Consider the function $f(x,y)=0$. What is its level curve for $c=0$?\\[1mm]
	}
	\uncover<5-> {
	In fact, you can have level curves that are very hard to visualize. For example :
	$$f(x,y) = 
	\begin{cases}
		1 & x \in \bQ \text{ and } y \in \bQ\\
		0 & \text{otherwise}
	\end{cases}$$
	for $c=0$ or $1$
	}
\end{frame}

\begin{frame}{Q2. (ii)}
	\begin{qsn}
        Describe the level curves and the contour lines for $f(x,y) = x^2+y^2$ corresponding to the values $c=-3,-2,-1,0,1,2,3,4$.
	\end{qsn}
    \uncover<2-> {
    \vspace*{1mm}
	 For $ c < 0$, the level curves are empty sets as $f(x,y) \geq 0 ~\forall (x,y) \in \mathbb{R}^2$\\[1mm]
    }
	\uncover<3-> {
	For $c = 0,$ the level curve is just the singleton set $\{(0,\;0)\}\subset \mathbb{R}^2$\\[1mm]
	}
	\uncover<4-> {
    For $c > 0,$ the level curve is the circle $\{(x,\;y)\in\mathbb{R}^2:x^2 + y^2 = c\}$\\[2mm]
    }
    \uncover<5-> {
    For a complete solution, you should describe the level curve $\mathcal{L}$ explicitly!\\
    }
    \uncover<6-> {
    For each value of $c$, you should also mention the contour line $\mathcal{C} = \mathcal{L}\times \{c\}$
    }
\end{frame}

\begin{frame}{Q2. (iii)}
    \begin{qsn}
        Describe the level curves and the contour lines for $f(x,y) = xy$ corresponding to the values $c=-3,-2,-1,0,1,2,3,4$.
	\end{qsn}
	\uncover<2-> {
	\vspace*{1mm}
	 For $c < 0$, level curves are rectangular hyperbolas $xy = c$ in the xy-plane with branches in the second and fourth quadrant. \\[1mm]
    }
	\uncover<3-> {
    For $c > 0$, level curves are rectangular hyperbolas $xy = c$ in the xy-plane with branches in the first and third quadrant. \\[1mm]
    }
	\uncover<4-> {
    For $c = 0$, the corresponding level curves are precisely the union of the x-axis and the y-axis
    }
\end{frame}

\subsection{Q4. Continuity of Function Combinations}

\begin{frame}{Recap}
\uncover<2-> {
\begin{df}[Euclidean norm]
Let $m\in \bN$. We define the euclidean distance between $x,y \in \bR^m$ by
$$\normtwo{x-y} = \sqrt{\sum_{i=1}^m (x_i-y_i)^2}$$
where $x=(x_1,x_2,\dots,x_m)$ and $y_m = (y_1,y_2,\dots,y_m)$
\end{df}
}
\end{frame}

\begin{frame}{Recap}
\uncover<2-> {
\begin{df}[Convergence in $\bR^m$]
Let $(x_n)$ be a sequence in $\bR^m$. If $\exists x \in \bR^m$ such that $\forall \epsilon>0, \exists N \in \bN$ for which
$$n\geq N \implies \normtwo{x_n-x}<\epsilon$$
then we say $(x_n)$ converges to $x$ and write $x_n \to x$.
\end{df}
}
\end{frame}

\begin{frame}{\textit{Something Extra}}
\uncover<2-> {
\begin{df}[Metric Spaces]
A set $X$ along with a function $d:X\times X \to \bR$ is called a metric space if the distance function $d$ satisfies the following :
\begin{enumerate}
\item $\forall x,y\in X, d(x,y)\geq 0$ and $d(x,y)=0 \Leftrightarrow x=y$
\item $\forall x,y\in X, d(x,y) = d(y,x)$
\item $\forall x,y,z\in X, d(x,y) + d(y,z) \geq d(z,x)$
\end{enumerate}
We will refer to $d(x,y) = d(y,x)$ as $\normtwo{x-y} = \normtwo{y-x}$
\end{df}
}
\end{frame}

\begin{frame}{\textit{Something Extra}}
\uncover<2-> {
\begin{df}[Convergence in General Metric Spaces]
Let $(x_n)$ be a sequence in a metric space $X$. If $\exists x \in X$ such that $\forall \epsilon>0, \exists N \in \bN$ for which $$n\geq N \implies \normtwo{x_n-x}<\epsilon$$
Then the sequence $(x_n)$ is said to converge to $x$ and we write $x_n \to x$\\
\end{df}
}
\end{frame}

\begin{frame}{Recap}
\uncover<2-> {
	The definition of continuity in functions from $\bR^2\to\bR$ or in general from any metric space $X\to\bR$ is parellel to that of functions from $\bR\to\bR$.\\
}
\uncover<3-> {
    Thus, one intuitively expects that sequential continuity would also hold and be equivalent to the definition of continuity. This is indeed the case as we show.
}
\uncover<4-> {
    \begin{thm}
    Let $X$ be a metric space, $x\in X$ and $f:X\to\bR$ be a function. Then, $f$ is continuous at $x$ iff $~\forall$ sequences $(x_n)$ such that $x_n \to x$, $f(x_n) \to f(x)$.
    \end{thm}
}
\end{frame}

\begin{frame}{Recap}
    \begin{proof}[Proof. (Forward)]
    \uncover<2-> {
    Let $f:X\to\bR$ be a continuous function and $(x_n)$ be a sequence in $X$ that converges to $x$.\\
    }
    \uncover<3-> {
    Given $\epsilon>0$, $\exists \delta>0$ such that $\forall y$ for which $\normtwo{y-x}<\delta$, $\norm{f(y)-f(x)}<\epsilon$\\
    }
    \uncover<4-> {
    Obtain this $\delta>0$, then $\exists N\in\bN$ such that $\forall n\geq N$, $\normtwo{x_n-x} < \delta$ (why?)\\
    }
    \uncover<5-> {
    Thus, $\forall n\geq N$, $\norm{f(x_n)-f(x)}<\epsilon$ and so $f(x_n)\to f(x)$.
    }
    \end{proof}
\end{frame}

\begin{frame}{Recap}
    \begin{proof}[Proof. (Backward)]
    \uncover<2-> {
    We will proceed via contrapositive. Let $f$ be a function that is not continuous at $x$.\\
    }
    \uncover<3-> {
    Thus, $\exists \epsilon>0$ such that $\forall \delta>0$, $\exists y$ for which $\normtwo{y-x}<\delta$ but $\norm{f(y)-f(x)}\geq \epsilon$.\\
    }
    \uncover<4-> {
    Construct a sequence $(x_n)$ by chosing $x_n$ to be such a $y$ for $\delta = \frac{1}{n}$.\\
    }
    \uncover<5-> {
    It is easy to see $x_n \to x$ (why?). Also, $f(x_n) \not\to f(x)$ (why?)\\
    }
    \uncover<6-> {
    Thus, we are done proving the contrapositive
    }
    \end{proof}
\end{frame}

\begin{frame}{Q4}
	\begin{qsn}
    Suppose $f, g : \mathbb{R} \to \mathbb{R}$ are continuous functions. Show that each of the following functions of
    $(x, y) \in \mathbb{R}^2$ are continuous:
    \begin{enumerate}
    	\item $f(x) \pm g(y)$
    	\item $f(x)g(y)$
    	\item $max\{f(x), g(y)\}$
    	\item $min\{f(x), g(y)\}$
    \end{enumerate}
    \end{qsn}
\end{frame}

\begin{frame}{Q4. (i)}
\begin{proof}
    \uncover<2-> {
        Let $(x_n,y_n)$ be any sequence such that $(x_n,y_n) \to (x,y)$\\[1mm]
    }
    \uncover<3-> {
        Thus, $x_n \to x$ and $y_n \to y$\\[1mm]
    }
    \uncover<4-> {
        Furthermore, by continuity of $f$ and $g$, $f(x_n) \to f(x)$ and $g(y_n) \to g(y)$\\[1mm]
    }
    \uncover<5-> {
        And by the theorem for sequences, $f(x_n)\pm g(y_n) \to f(x)\pm g(y)$\\[1mm]
    }
    \uncover<6-> {
        Since $(x_n,y_n)$ was arbitrary, we are done.
    }
\end{proof}
\end{frame}

\begin{frame}{Q4. (ii)}
\begin{proof}
    \uncover<2-> {
        Let $(x_n,y_n)$ be any sequence such that $(x_n,y_n) \to (x,y)$\\[1mm]
    }
    \uncover<3-> {
        Thus, $x_n \to x$ and $y_n \to y$\\[1mm]
    }
    \uncover<4-> {
        Furthermore, by continuity of $f$ and $g$, $f(x_n) \to f(x)$ and $g(y_n) \to g(y)$\\[1mm]
    }
    \uncover<5-> {
        And by the theorem for sequences, $f(x_n)g(y_n) \to f(x)g(y)$\\[1mm]
    }
    \uncover<6-> {
        Since $(x_n,y_n)$ was arbitrary, we are done.
    }
\end{proof}
\end{frame}

\begin{frame}{Q4. (iii)}
\begin{proof}
    \uncover<2-> {
        Let $(x_n,y_n)$ be any sequence such that $(x_n,y_n) \to (x,y)$\\[1mm]
    }
    \uncover<3-> {
        Thus, $x_n \to x$ and $y_n \to y$\\[1mm]
    }
    \uncover<4-> {
        Furthermore, by continuity of $f$ and $g$, $f(x_n) \to f(x)$ and $g(y_n) \to g(y)$\\[1mm]
    }
    \uncover<5-> {
        And by the theorem for sequences, $f(x_n)\pm g(y_n) \to f(x)\pm g(y)$\\[1mm]
    }
    \uncover<6-> {
        Thus, we also have $\norm{f(x_n)-g(y_n)} \to \norm{f(x)-g(y)}$ (recall convergence theorem for $\norm{x_n}$)\\[1mm]
    }
    \uncover<7-> {
        Now recall $\max\{a,b\} = \frac{(a+b)+|a-b|}{2}$. Hence $\max\{f(x_n),g(y_n)\} \to \max\{f(x),g(y)\}$\\[1mm]
    }
    \uncover<8-> {
        Since $(x_n,y_n)$ was arbitrary, we are done.
    }
\end{proof}
\end{frame}

\begin{frame}{Q4. (iv)}
\uncover<2-> {
    Yeah, this is getting boring, let us skip!
}
\end{frame}

\subsection{Q6. Partial Derivatives at 0}
\begin{frame}{Q6. (ii)}
    \begin{qsn}
        Examine the function given by
         $$f(x,y) = 
         \begin{cases}
         0 & \text{where } (x,y)=(0,0)\\
         \frac{\sin^2(x+y)}{|x|+|y|} & \text{otherwise}
         \end{cases}$$
        for the existence of partial derivatives at $(0, 0)$.
    \end{qsn}
\end{frame}

\begin{frame}{Q6. (ii)}
    \uncover<2-> {
	    $$f_x(0,\;0) = \displaystyle\lim_{h\to 0}\frac{f(0+h,\;0) - f(0,\;0)}{h} = \displaystyle\lim_{h\to 0}\left(\frac{\sin^2(h)}{h|h|}\right)$$
	}
    \uncover<3-> {
    This limit does not exist (why?)\\[1mm]
	}
	\uncover<4-> {
	Hint : Consider the two sequences with $n^{th}$ term given by $1/n$ and $-1/n$.\\[1mm] 
	}
    \uncover<5-> {
    Observe that $f$ is symmetric in $x$ and $y$, hence $f_y(0,\;0)$ also does not exist.
    }
\end{frame}

\subsection{Q8. Continuity $\nRightarrow$ existence of partial derivatives}
\begin{frame}{Q8}
    \begin{qsn}
        Let $f(0,0) = 0$ and 
        \begin{equation*}
            f(x,y) = 
            \begin{cases}
                    x\sin(1/x) + y\sin(1/y) & if~x\neq 0,\; y \ne 0\\
                    x \sin(1/x) & if~x\neq 0,\; y = 0\\
                    y \sin(1/y) & if~x= 0,\; y \neq 0\\
            \end{cases}
        \end{equation*}
        Show that none of the partial derivatives of $f$ exist at $(0, 0)$ although $f$ is continuous at $(0, 0)$.
    \end{qsn}
\end{frame}
\begin{frame}{Q8}
    \textbf{Claim} : $f$ is continuous at $(0,0)$\\
    \pause
    \begin{proof}
    We only need to prove that $\lim\limits_{(x,y) \to (0,0)} f(x,y) = 0$\\
    \pause
    Given any $\epsilon > 0$, $\exists\delta > 0$ such that $0<|x|<\delta\implies |x\sin(1/x)| < \epsilon/2$. (why?)\\
    \pause
    Also note that $\norm{x}\leq\normtwo{(x,y)}$ and $\norm{y}\leq \normtwo{(x,y)}$\\
    \pause
    Let $0<\normtwo{(x,y)}<\delta$, then
    \begin{itemize}
        \item if $x=0$, $\norm{f(x,y)} = \norm{y\sin(1/y)}<\epsilon/2<\epsilon$
        \item if $y=0$, $\norm{f(x,y)} = \norm{x\sin(1/x)}<\epsilon/2<\epsilon$
        \item otherwise, $\norm{f(x,y)} = \norm{x\sin(1/x)+y\sin(1/y)} \leq \norm{x\sin(1/x)}+\norm{y\sin(1/y)} <\epsilon$
    \end{itemize}
    \pause
     Clearly, $\norm{f(x,y)} < \epsilon$.\\
    \pause
    Thus, $0<\normtwo{(x,y)-(0,0)}<\delta \implies \norm{f(x,y)-f(0,0)}<\epsilon$ and the proof is complete.
    \end{proof}
\end{frame}

\begin{frame}{Q8}
	Now let us show that the partial derivatives don't exist.\\
	\pause
	$$f_x(0,\;0) = \lim_{h\to 0}\frac{f(0 + h,\; 0) - f(0,\;0)}{h} = \lim_{h\to 0}\sin\left(\frac{1}{h}\right)$$
	\pause
	Which we know, does not exist\\[1mm]
	\pause
	Again, note that $f$ is symmetric in $x$ and $y$. Hence, the other partial derivative does not exist.
\end{frame}

\subsection{Q10. Existence of every directional derivative $\nRightarrow$ differentiability}

\begin{frame}{Q10}
   \begin{qsn}
       Let $f(x,y) = 0$ if $y=0$ and 
       $$
        f(x,y) = \frac{y}{|y|} \sqrt{x^2 + y^2}~if~y \ne 0
       $$
       Show that $f$ is continuous at $(0, 0)$, $D_uf(0, 0)$ exists for every vector u, yet f is not differentiable at $(0, 0)$.
   \end{qsn}
\end{frame}

\begin{frame}{Q10}
\begin{proof}[Proof. (Continuity)]
    Continuity of $f$ follows easily from $\epsilon-\delta$ condition\\
    \uncover<2->{
	$|f(x, y) - f(0, 0)| = \left|\sqrt{x^2 + y^2}\right|$ for $y \neq 0$ and $|f(x, y) - f(0, 0)| = 0$ for $y = 0$\\
	}
	\uncover<3-> {
	Thus, in general, we have that $|f(x, y) - f(0, 0)| \leq \left|\sqrt{x^2 + y^2}\right| = \normtwo{(x,y)}$ \\
	}
	\uncover<4-> {
	Let $\delta := \epsilon$ and we are done.
	}
\end{proof}
\end{frame}
\begin{frame}{Q10}
\begin{proof}[Proof. (Directional Derivatives)]
    \uncover<2->{
    For a unit vector $\textbf{u} := (u_1, u_2)$ and $t \neq 0,$
    }
    \uncover<3-> {
	\[\frac{f\left(0+t u_{1}, 0+t u_{2}\right)-f(0,0)}{t} = \left\{
		\begin{array}{c c}
			0 & u_2 = 0\\
			\frac{u_2}{|u_2|} & u_2 \neq 0
		\end{array}
	\right.\]
	}
	\uncover<4-> {
	Hence, $\left(\mathbf{D_u} f\right)(0,0)$ exists for all $\textbf{u}.$ Thus, all directional derivatives exist
    }
\end{proof}
\end{frame}
\begin{frame}{Q10}
    \uncover<2->{
	For $f$ to be differentiable, we must check whether
	\[\lim_{(h, k) \rightarrow(0,0)} \frac{f\left(0+h, 0+k\right)-f\left(0, 0\right)-f_x(0,0) h- f_y(0,0) k}{\sqrt{h^{2}+k^{2}}}=0\]
	}
	\uncover<3->{
	For $(h,k) \neq (0,0),$ we have that
	\[\frac{f\left(0+h, 0+k\right)-f\left(0, 0\right)-0 h- 1k}{\sqrt{h^{2}+k^{2}}} = \frac{k}{|k|} - \dfrac{k}{\sqrt{h^2 + k^2}}\]
	}
	\uncover<4-> {
	Consider the sequence $((h_n,k_n))$ given by $h_n = k_n = \frac{1}{n}$.\\
	Obviously $(h_n,k_n)\to(0,0)$ but
	}
	\uncover<5-> {
	$$\frac{k_n}{|k_n|} - \dfrac{k_n}{\sqrt{h_n^2 + k_n^2}} = 1-\frac{1}{\sqrt{2}}$$
	}
	\uncover<6-> {
	Which clearly does not converge to 0.
	}
\end{frame}
\end{document}